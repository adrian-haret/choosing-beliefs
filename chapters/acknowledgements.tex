Writing can be an excruciating affair, 
partly because one is forced to confront the 
gap between what is in one's head and what comes out on the page,
and partly because \LaTeX is so tedious.
So the fact that this thesis managed to get done is proof 
that many stars were aligned, and many thanks are in order.

Most of the text to follow was written during January--April, 2020,
and the influence of events unfolding across the world 
can be felt in the tone of the examples: 
from examples about Oscar nominees, conceived at the beginning of the year,
to examples about doctors who have to decide on treatments for novel diseases, added in March.
Conditions like these can be rattling at the best of times.
In my case, they brought back memories of the Otto Wagner Spital (OWS) in Vienna, 
where I spent a few weeks in June, 2019. It is to the doctors and staff of the OWS
that I want to extend my first thanks and gratitude.

The thesis itself grew out of successful collaborations and careful guidance.
First acknowledgments go to my advisor, Stefan Woltran, who took a chance 
on me some years ago and hired me as a research assistant in his project, 
then guided me along the winding footpaths of academic research.
Along the way there was help and encouragement: in the early days from Stefan R\"ummele;
in the latter years from Johannes Wallner, who so finely took up the role of 
employer, inspiration and colleague; and, not least, from Thomas Eiter, 
my second advisor.

Equally important were all the co-authors and collaborators I was fortunate to work with,
listed here in chronological order:
Thomas Linsbichler, Martin Diller, Jean-Guy Mailly, J\'er\^ome Delobelle, S\'ebastien Konieczny,
Julien Rossit, Andreas Pfandler, Nadia Creignou, Odile Papini and Martin Lackner.  
Thanks go out to the OeAD for funding my research stay at Univ. Paris-Dauphine,
in Paris, during February--July, 2018, through a Marietta Blau grant. 
It was in Paris that I got my first taste of working on social choice topics, 
with Umberto Grandi, Arianna Novaro, Meltem \"Ozt\"urk, Stefano Moretti and Hossein Khani.
Special thanks go to J\'er\^ome Lang, who hosted me in Paris; 
to Umberto, who hosted me in Toulouse; 
and to Nadia and Odile, who hosted me in Marseille.

Key in all this was the kindness and expertise of so many dedicated people,
whose commitment to making things run smoothly was frankly invaluable:
first and foremost are Juliane, Eva and Beatrix, along with the entire 
team behind the LogiCS program; then there is the team at the OeAD 
that coaches students in how to apply for their grants; and Eleni, at Dauphine.

Over the years I have been lucky to spend time with many splendid people,
important for me in ways I cannot, and do not want to quantify, but still wish 
to acknowledge:
Anna, Jan, Martin, Tobi, Thomas, Wolfgang and Zeynep, in Vienna, 
without whom lunches would have not been the same; 
my cousin Raluca, whose spare room in Vincennes I occupied in early 2018;
Audrey, Emma and Harold, whose spare room in the 9$^{\text{th}}$ arrondissement in Paris
I occupied in mid-2018; 
Arianna and Dennis, in Toulouse;
Hossein, at Dauphine.

Lastly, my gratitude goes out to Cătălin, a man for all seasons;
my parents, who were there throughout it all;
and Renate, who was there during the most difficult moments.