Belief change, in the AGM tradition, 
gathers under a common methodological umbrella
an array of operations, covering both single-agent and multi-agent processes. 
These operations are linked by the use of propositional logic as a \emph{lingua franca},
are related by a network of interconnected rationality constraints
and are united by the idea that they all describe, 
in some way or another, the dynamics of beliefs and information. 

In this thesis we want to see belief change, thus construed,
as akin to making a decision: according to this perspective an agent,
or group of agents, faced with new information must make a decision 
as to what part of the new information to adopt, 
in a manner that balances both the agents' own positions,
as well as certain rationality constraints, e.g., consistency. 
The parallel between changing a belief and making a decision
is encouraged by the observation that 
both areas use the same underlying mechanisms of choices and preferences 
to rationalize their operations, in the process employing 
strikingly similar rationality constraints.

Though we are not the first to make this observation, 
we argue that there is still space to explore its implications.
Seeing belief change operators as choice procedures that rely on 
preferences over outcomes allows us to tap both a series of useful intuitions 
about what belief change operators do,
and a set of properties, scattered throughout the rational and social 
choice literature, that can aid the design of new instruments for belief change.

Thus, one side to our contribution to this thesis 
revolves around three existing prominent belief change operations,
i.e., revision, update and merging, where these insights are employed
in order to expand their range of application.
For the case of revision we propose new postulates 
that deal with the way in which the prior information influences the 
revision process. 
For merging, which is a multi-agent operation, we adapt properties 
from the social choice literature, such as strategyproofness 
and proportionality, that formalize various aspects of fairness.
We also look towards applications of revision and update to 
the Horn fragment of propositional logic, and extend existing work
by studying weaker variants of the traditional postulates 
used in these cases.
Another part of our contribution consists of a new type 
of belief change operation, which we call enforcement, 
and which we put forward both as a belief change operation 
in its own right and as theoretical 
background for a model of preference revision.

In all these cases we use the usual belief change tools 
of logical postulates, which we reinterpret,
as part of our choice perspective, as
constraints on a choice function over possible outcomes.
Through a set of representation results we are able to show, then,
that the postulates can be rationalized in the familiar way using 
rankings on possible outcomes.