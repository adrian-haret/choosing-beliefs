Ans\"atze der Wissensadaption die der AGM Schule folgen bilden eine wesentliche Grundlage um eine Vielzahl von Operationen, im speziellen in Bereichen in denen Agenten miteinander agieren, zu studieren.  Bei der Untersuchung dieser Operatoren dient Aussagenlogik als lingua franca, nicht zuletzt um Bedingungen an diese Operatoren zu spezifieren, welche es erlauben rationale Operatoren von irrationalen zu unterscheiden. Aufgrund unterschiedlicher Auffassungen von Rationalit\"at ergeben sich dadurch auch verschiedene und vernetzte Definitionen von Aspekten von Rationalit\"at, die diese Verschiedenartigkeit widerspiegeln.

In dieser Disseration vertreten wir die These, dass Wissensadaption eine Verwandtschaft mit dem allgemeinen Problem der Entscheidungsfindung aufweist. Insbesonders muss ein Agent, oder eine Gruppe von Agenten, die Entscheidung treffen wie Wissen adaptiert werden sollte, wenn neue Informationen vorhanden sind und dabei sowohl die eigenen Positionen als auch die angesprochenen Bedingungen an Rationalit\"at ber\"ucksichtigen. Beispielsweise sollte das Resultat der Adaption konsistent sein.  Die Parallele zwischen Wissensadaption und Theorien zur Entscheidungsfindung wird durch den Gedankengang gest\"arkt, dass die jeweiligen Forschungsgebiete die gleichen Mechanismen f\"ur eine Auswahl von Optionen verwenden. Insbesonders werden gleiche Ans\"atze verwendet um zu zeigen, dass man die jeweiligen Operationen der Gebiete als rational ansehen kann.

Obwohl die Verbindung zwischen Wissensadaption und Entscheidungstheorie schon fr\"uher aufgezeigt wurde, argumentieren wir, dass noch viele Grundlagen offen sind. So können Operatoren der Wissensadaption als Entscheidungsprozesse angesehen werden, indem man die möglichen Resultate einer Adaption nach Pr\"aferenzen reiht. Durch diese Sichtweise können nicht nur Intuitionen die solchen Operatoren zugrunde liegen sichtbar gemacht werden, es können auch verschiedene Eigenschaften von verwandten Gebieten, die sich mit Theorien von rationalen Entscheidungen und der Sozialwahltheorie besch\"aftigen, auf Operationen der Wissensadaption angewendet werden. Ergebnisse solcher Untersuchungen f\"uhren, unter anderem, zu neuen Operationen in der Wissensadaption, welche auf diesen Eigenschaften beruhen.

Die wissenschaftlichen Beitr\"age dieser Dissertation sind zum einen eine Erweiterung der Studie von drei prominenten Familien von Operationen in der Wissensadaption: Revision, Update und Vereinigung von Wissen. F\"ur alle drei Familien erweitern wir deren Anwendungsspektrum durch die oben genannten Ans\"atze. F\"ur Wissensrevision schlagen wir neue Postulate vor, die sich damit besch\"aftigen, wie bereits vorhandenes Wissen die Revision beinflusst. F\"ur Wissensvereinigung, welche sich mit der Integration von Wissen im Kontext von Gruppen von Agenten besch\"aftigt, n\"utzen wir Eigenschaften aus der Sozialwahltheorie, insbesondere solche Eigenschaften die sich damit befassen ob eine Operation manipuliert werden kann und solche die sich auf des Konzept der Proportionalit\"at beziehen. Weiters erweitern wir vorangegangene Arbeiten im Bereich der Revision und des Updates von Wisssen welches in Hornlogik, einem Fragment der Aussagenlogik, formuliert ist. Wir sehen uns hierf\"ur schw\"achere Varianten der g\"angigen Postulate an. Zum anderen beinhaltet die Disseration eine Untersuchung einer neuen Familie von Wissensadaptionoperationen, welche wir Enforcement nennen und als Grundlage f\"ur Revision von Pr\"aferenzen heranziehen.

In all unseren wissenschaftlichen Beitr\"agen verfolgen wir die in der Wissensadaption \"ubliche Postulat-basierte Methodologie, um, im Einklang mit der oben genannten Verwandtschaft zur Entscheidungstheorie, diese Postulate als Wahlprozesse, sprich als Auswahlfunktionen von möglichen Resultaten, zu interpretieren. Durch Repr\"asentationsresultate zeigen wir, dass die Postulate durch Verwendung von Ordnungen rationalisiert werden können.